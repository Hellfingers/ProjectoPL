%
% Layout retirado de http://www.di.uminho.pt/~prh/curplc09.html#notas
%
\documentclass{report}
\usepackage[portuges]{babel}
\usepackage[utf8]{inputenc}
\usepackage{url}
%\usepackage{alltt}
%\usepackage{fancyvrb}
\usepackage{listings}
%LISTING - GENERAL
\lstset{
	basicstyle=\small,
	numbers=left,
	numberstyle=\tiny,
	numbersep=5pt,
	breaklines=true,
    frame=tB,
	mathescape=true,
	escapeinside={(*@}{@*)}
}
%
%\lstset{ %
%	language=Java,							% choose the language of the code
%	basicstyle=\ttfamily\footnotesize,		% the size of the fonts that are used for the code
%	keywordstyle=\bfseries,					% set the keyword style
%	%numbers=left,							% where to put the line-numbers
%	numberstyle=\scriptsize,				% the size of the fonts that are used for the line-numbers
%	stepnumber=2,							% the step between two line-numbers. If it's 1 each line
%											% will be numbered
%	numbersep=5pt,							% how far the line-numbers are from the code
%	backgroundcolor=\color{white},			% choose the background color. You must add \usepackage{color}
%	showspaces=false,						% show spaces adding particular underscores
%	showstringspaces=false,					% underline spaces within strings
%	showtabs=false,							% show tabs within strings adding particular underscores
%	frame=none,								% adds a frame around the code
%	%abovecaptionskip=-.8em,
%	%belowcaptionskip=.7em,
%	tabsize=2,								% sets default tabsize to 2 spaces
%	captionpos=b,							% sets the caption-position to bottom
%	breaklines=true,						% sets automatic line breaking
%	breakatwhitespace=false,				% sets if automatic breaks should only happen at whitespace
%	title=\lstname,							% show the filename of files included with \lstinputlisting;
%											% also try caption instead of title
%	escapeinside={\%*}{*)},					% if you want to add a comment within your code
%	morekeywords={*,...}					% if you want to add more keywords to the set
%}

\usepackage{xspace}

\parindent=0pt
\parskip=2pt

\def\darius{\textsf{Darius}\xspace}
\def\java{\texttt{Java}\xspace}

\title{Processamento de Linguagens (3º ano de Curso)\\ \textbf{Trabalho Prático 2}\\ Relatório de Desenvolvimento}
\author{Jorge Miguel Sol Ferreira (a64293) \and Pedro José Freitas da Cunha (a67677) \and José Pedro Brito Pereira (a67680)\\Grupo 35 }
\date{\today}

\begin{document}

\maketitle

\begin{abstract}
Este relatório documentará todos os passos tomados na realização do segundo trabalho prático da Unidade Curricular de Processamento de Linguagens.\\ Neste projecto é requerida a implementação de um compilador de uma Linguagem de Programação Imperativa Simples e posteriormente gerador de código assembly para uma máquina de stacks virtual.
\end {abstract}
\tableofcontents

\chapter{Introdução} \label{intro}

\begin{description}
  \item [Enquadramento] \textbf{bla bla} bla bla
  \item [Conteúdo do documento] \textsf{ble ble} \texttt{ble} ble
  \item [Resultados -- pontos a evidenciar] \emph{bli bli bli bli}
  \item [Estrutura do documento] \underline{blo blo blo}
\end{description}



\section*{Estrutura do Relatório} \
explicar como está organizado o documento, referindo os capítulos existentes
e a sua articulação explicando o conteúdo de cada um.
No capítulo de Análise e Especificaçao faz-se uma análise detalhada do problema proposto
de modo a poder-se especificar  as entradas, resultados e formas de transformação.\\
etc. \ldots\\
No capítulo de Conclusões termina-se o relatório com uma síntese do que foi dito,
as conclusões e o trabalho futuro

\chapter{Análise e Especificação} 
\section{Descrição informal do problema}
\section{Especificação do Requisitos}
\subsection{Dados}
\subsection{Pedidos}
\subsection{Relações}

\chapter{Concepção/desenho da Resolução}
\section{Estruturas de Dados}
\section{Algoritmos}

\chapter{Codificação e Testes}
\section{Alternativas, Decisões e Problemas de Implementação}
\section{Testes realizados e Resultados}
Mostram-se a seguir alguns testes feitos (valores introduzidos) e
os respectivos resultados obtidos:

%\VerbatimInput{teste1.txt}


\chapter{Conclusão}
Síntese do Documento.\\
Estado final do projecto; Análise crítica dos resultados.\\
Trabalho futuro.

\appendix
\chapter{Código do Programa}


\begin{verbatim}
      aqui deve aparecer o código do programa,
      tal como está formato no ficheiro-fonte "darius.java"
\end{verbatim}

















\end{document} 