%
% Layout retirado de http://www.di.uminho.pt/~prh/curplc09.html#notas
%
\documentclass{report}
\usepackage[portuges]{babel}
\usepackage[utf8]{inputenc}
\usepackage{url}
\usepackage{alltt}
\usepackage{fancyvrb}
\usepackage{listings}

%LISTING - GENERAL
\lstset{
	basicstyle=\small,
	numbers=left,
	numberstyle=\tiny,
	numbersep=5pt,
	breaklines=true,
    frame=tB,
	mathescape=true,
	escapeinside={(*@}{@*)}
}

\usepackage{xspace}

\parindent=0pt
\parskip=2pt

\def\darius{\textsf{Darius}\xspace}
\def\java{\texttt{Java}\xspace}

\title{Processamento de Linguagens (3º ano de Curso)\\ \textbf{Trabalho Prático 2}\\ Relatório de Desenvolvimento}
\author{Jorge Miguel Sol Ferreira (a64293) \and Pedro José Freitas da Cunha (a67677) \and José Pedro Brito Pereira (a67680)\\Grupo 35 }
\date{\today}

\begin{document}

\maketitle

\begin{abstract}
Este relatório documentará todos os passos tomados na realização do segundo trabalho prático da Unidade Curricular de Processamento de Linguagens.\\ Neste projecto é requerida a implementação de um compilador de uma Linguagem de Programação Imperativa Simples e posteriormente gerador de código assembly para uma máquina de stacks virtual.
\end {abstract}
\tableofcontents

\chapter{Introdução} \label{intro}

\begin{description}
  \item [Enquadramento] Um \textbf{compilador} é uma peça de software que transforma o código fonte numa dada linguagem de alto nível em instruções que a máquina entenda (Código Máquina). As fazes da compilação incluem:\\Análise léxica\\Análise Sintática\\Análise Semântica\\Geração de Código
  \item [Conteúdo do documento] Neste documento encontrar-se-ão as fases de resolução do problema especificado
  \item [Resultados -- pontos a evidenciar] O resultado do projecto a desenvolver será um gerador de códigoassembly para uma máquina de stacks virtual, partindo de uma Linguagem de Programação Imperativa Simples.
\end{description}



\section*{Estrutura do Relatório} \
No capítulo \ref{analiseEsp} iremos apresentar o caso de estudo em causa. No capítulo \ref{concepcao} iremos apresentar a estrutura de dados auxiliar à análise semântica e sua utilização no compilador a desenvolver, bem como fazer um esboço do que queremos que seja a nossa linguagem de programação (Gramática Independente do Contexto), para além de alguns exemplos de frases que tenham erros sintáticos, semânticos e frases correctas segundo a nossa especificação. No capítulo \ref{code} iremos apresentar os passos utilizados para geração de código, se possível ou, em alternativa notificaçao de erro sintático. Finalmente, no capítulo \ref{conc} faremos uma apreciação crítica do trabalho realizado e trabalhos futuros. Em anexo iremos colocar o código desenvolvido que permitirá a geração de código máquina.

\chapter{Análise e Especificação} \label{analiseEsp}

\section{Descrição informal do problema}
Neste projecto é pretendido o desenho de uma Linguagem de Programação Imperativa Simples, para de seguida criar um compilador que gere pseudo-código Assembly de uma Máquina Virtual de Stacks
\section{Especificação do Requisitos}
Os requisitos para o compilador/linguagem a implementar são os seguintes:
\begin{itemize}
  \item Permitir manusear variáveis do tipo inteiro(escalar ou array) e do tipo string.
  \item Realizar as seguintes operações:
	\begin{itemize}
	\item Atribuições de expressões a variáveis.
	\item Ler do Standard Input.
	\item Escrever para o Standard Output.
	\end{itemize}
 \item Ciclos(for, while) e instruções Condicionais(if..else).
 \item Operações Aritméticas, Relacionais e Lógicas sobre inteiros.
 \item Indexação sobre arrays.
 \item As declarações de variáveis deverão ser no início do programa.
 \item Não deverá ser possível realizar redeclaraçoes nem utilizações sem declaraçao prévia.
 \item Se não existirem atribuições a uma variável, o valor da mesma deverá ser indefinido
\end{itemize}

\chapter{Concepção/desenho da Resolução} \label{concepcao}
\section{Gramática}
Ao desenvolver a gramática tentámos fazer com que a mesma ficasse o mais próximo possível do C, funcionando como uma versão simplificada do C.\\Abaixo encontra-se a gramática desenhada para a Linguagem:

\begin{verbatim}
Prog    ->  Declss Instrs

Declss  ->  Decls

Decls   ->  InitVar
        |   Decls InitVar

InitVar ->  INT Var '[' num ']' ';'
        |   INT Var ';'
        |   STRING Var ';'

Var     ->  id

Instrs  ->  Instr
        |   Instrs Instr

Instr   ->  If
        |   While
        |   For
        |   Atr ';'
        |   IO ';'

If      ->  IF '(' Cond ')' '{' Instrs '}' Else

Else    ->  &
        |   ELSE '{' Instrs '}'

While   ->  WHILE '(' Cond ')' '{' Instrs '}'

For     ->  FOR '(' Atr ';' Cond ';' Atr ')' '{' Instrs '}'

IO      ->  PRINT Out
        |   INPUT Var
        |   INPUT Var '[' Exp ']'

Out     ->  Exp
        |   str

Atr     ->  Var '=' Exp
        |   Var '[' Exp ']' '=' Exp

Exp     ->  Termo
        |   Exp OpA Termo

Termo   ->  Fator
        |   Termo OpM Fator

Fator   ->  Var
        |   Var '[' Exp ']'
        |   num
        |   str
        |   '(' Exp ')'
        |   '!' Exp

Cond    :   Comp
        |   '(' Cond ')'
        |   Cond '&''&' Cond
        |   Cond '|''|' Cond

Comp    :   Exp
        |   Exp OpComp Exp
\end{verbatim}

\section{Estruturas de Dados}
Para realizarmos a análise semântica temos uma tabela de Hash para guardar todos os identificadores de variáveis e seus tipos de dados. \\A estrutura de dados da tabela de Hash é a seguinte:\\
\begin{verbatim}
struct list{
           char *key;
           char *type;
           int init;
           int ind;
           struct list* next;};

struct table{
           int size;
           int elems;
           struct list **list;
};
\end{verbatim}
Com a estrutura acima referida podemos então informações sobre uma variável (Identificador, Tipo, Estado de Inicialização e Índice do Registo na Máquina).

\section{Exemplos de Programas da Linguagem}
Abaixo irão estar apresentados vários programas, em que os dois primeiros não cumprem os requisitos sintáticos (\ref{esint}) e semânticos (\ref{esem}) da nossa linguagem. Posteriormente iremos apresentar um programa que esteja sintática e semanticamente correcto (\ref{pc}).
\subsection{Programas com erros sintáticos}\label{esint}
Programa 1:\\
O programa abaixo quebra uma das regras estipuladas para a Linguagem de Programação: "As declarações de variáveis deverão ser no início do programa."
\begin{verbatim}
int a;
int b;
for(b=0;b<10;b=b+1)
     int c=b;
print a;
\end{verbatim}
\subsection{Programas com erros semânticos}\label{esem}
Programa 1:\\
O programa que se segue quebra uma das especificações da linguagem: " Não deverá ser possível utilizações de variáveis sem declaração prévia."

\begin{verbatim}
int a;
int b;
for(a=0;a<10;a=a+1)
      for(b=10;b>0;b=b-1)
            c=c+a+b;
print a;
\end{verbatim}
\subsection{Programas correctos}\label{pc}
Programa 1:\\
O programa abaixo segue as regras sintáticas e semânticas:
\begin{verbatim}
int n;
int first;
int second;
int next;
int i;
 
first=0;
second=1;
 
print "Insira o numero de termos";
input n;
 
for(i=0;i<n;i=i+1){
        if(i<=1){
                next=i;
        }
        else{
                next = first + second;
                first = second;
                second = next;
        }
        print next;
}
\end{verbatim}

\chapter{Codificação e Testes} \label {code}
\section{Decisões de Implementação}
Ao decorrer do desenvolvimento do gerador de código assembly um dos desafios que foi necessário resolver foi a criação de labels para controlar os saltos que as instruções condicionais necessitam. Para resolver essa situação criámos uma stack que servirá para guardar os números das labels necessárias. Por cada \emph{if sem else} será adicionada uma label, por cada \emph{if com else}, \emph{while}, \emph{for} serão adicionadas duas labels à stack. Por questão de simplicidade criámos um limite de 100 labels a criar.
\section{Testes realizados e Resultados}
Mostram-se a seguir alguns testes feitos (valores introduzidos) e
os respectivos resultados obtidos:\\
Para o exemplo encontrado em \ref{pc} o assembly gerado é o seguinte:
\begin{verbatim}
        PUSHI 0
        PUSHI 0
        PUSHI 0
        PUSHI 0
        PUSHI 0
        START
        PUSHI 0
        STOREG 1
        PUSHI 1
        STOREG 2
        PUSHS "Insira o numero de termos"
        WRITES
        READ
        ATOI
        STOREG 0
        PUSHI 0
        STOREG 4
L2:
        PUSHG 4
        PUSHG 0
        INF
        JZ L1
        JUMP L4
L3:
        PUSHG 4
        PUSHI 1
        ADD
        STOREG 4
        JUMP L2
L4:
        PUSHG 4
        PUSHI 1
        INFEQ
        JZ L5
        PUSHG 4
        STOREG 3
        JUMP L6
L5:
        PUSHG 1
        PUSHG 2
        ADD
        STOREG 3
        PUSHG 2
        STOREG 1
        PUSHG 3
        STOREG 2
L6:
        PUSHG 3
        WRITEI
        JUMP L3
L1:
        STOP
\end{verbatim}

Cálculo do menor elemento de um array:
Input:
\begin{verbatim}
int a[30];
int i;
int num;
int menor;

print "Insira o numero total de elementos:";
input num;

for(i=0;i<num;i=i+1){
    print "Insira um numero";
    input a[i];
}

menor = a[0];

for(i=0;i<num;i=i+1){
    if(a[i]<menor){
        menor = a[i];
    }
}

print "O menor elemento é o: ";
print menor;
\end{verbatim}
Output:
\begin{verbatim}
        PUSHN 30
        PUSHI 0
        PUSHI 0
        PUSHI 0
        START
        PUSHS "Insira o numero total de elementos:"
        WRITES
        READ
        ATOI
        STOREG 2
        PUSHI 0
        STOREG 1
L2:
        PUSHG 1
        PUSHG 2
        INF
        JZ L1
        JUMP L4
L3:
        PUSHG 1
        PUSHI 1
        ADD
        STOREG 1
        JUMP L2
L4:
        PUSHS "Insira um numero"
        WRITES
        PUSHG 1
        PUSHG 0
        READ
        ATOI
        STOREN
        JUMP L3
L1:
        PUSHI 0
        PUSHG 0
        LOADN
        STOREG 3
        PUSHI 0
        STOREG 1
L6:
        PUSHG 1
        PUSHG 2
        INF
        JZ L5
        JUMP L8
L7:
        PUSHG 1
        PUSHI 1
        ADD
        STOREG 1
        JUMP L6
L8:
        PUSHG 1
        PUSHG 0
        LOADN
        PUSHG 3
        INF
        JZ L9
        PUSHG 1
        PUSHG 0
        LOADN
        STOREG 3
L9:
        JUMP L7
L5:
        PUSHS "O menor elemento é o: "
        WRITES
        PUSHG 3
        WRITEI
        STOP

\end{verbatim}

\chapter{Conclusão} \label {conc}
Um compilador é uma peça de software complexa que tem uma tarefa crítica na geração de executáveis. Todas as instruções em código máquina deverão estar correctas e deverá haver pouquíssima tolerância a erros, já que a falha em alguma das instruçoes poderá ter consequências catastróficas.\\Acima foi apresentada a gramática, decisões tomadas na geração de código máquina e exemplos de programas correctos segundo a mesma. No apêndice deste documento iremos apresentar o código utilizado na resolução do projecto.\\
Neste momento tempos desenvolvido um gerador de código assembly que gera correctamente código máquina, trabalhando de momento com inteiros, arrays de inteiros e Strings. O facto da stack de labels ter uma limitação de 100 labels pode ser limitador para gerar código para programas de grande dimensão. Ainda será preciso adicionar bastantes funcionalidades a este compilador como suporte a funções e reconhecimento de mais tipos de dados.\\
Para aprimorar os resultados obtidos neste projecto os seguntes pontos deverão ser resolvidos:\\
\begin{itemize}
	\item Aumentar a capacidade da stack de labels, ou possivelmente torná-la dinâmica.
	\item Adicionar o suporte a funções e tipos de dados adicionais.
\end{itemize}

\appendix
\chapter{Código do Programa}
Código do Analisador Léxico para reconhecer os símbolos terminais:
\begin{verbatim}
%{
    #include <stdlib.h>
%}
%x COMENTARIO

%option yylineno

num [0-9]+
pal [a-zA-Z]+

%%

\/\/.*          {
                    ;
                }

"int"             { 
                    return INT; 
                }

"string"         { 
                    return STRING; 
                }

"if"            {
                    return IF;
                }

"else"          {
                    return ELSE;
                }

"while"         {
                    return WHILE;
                }

"for"           {
                    return FOR;
                }

"print"         {
                    return PRINT;
                }

"input"         {
                    return INPUT;
                }

[+-]                {
                        yylval.valOp = yytext[0];
                        return OpA;
                    }

[*\/%]              {
                        yylval.valOp = yytext[0];
                        return OpM;
                    }

\"[^\"]*\"          {yylval.valc = strdup(yytext);return str;BEGIN INITIAL;}

(([<>][=]?)|"=="|"!=")      {
                                yylval.valc = strdup(yytext);
                                return OpComp;
                            }

([<>\(\)\{\}\[\];=!&]|"|")      {
                                return yytext[0]; 
                            }

{pal}           { 
                    yylval.valc = strdup(yytext);
                    return id;
                }

{num}           {
                    yylval.vali = atoi(yytext);
                    return num;
                }

.|\n            { 
                    ; 
                }

%%

int yywrap()
    { return(1); }

\end{verbatim}

Código do Analisador Sintático/Analisador Semântico/Gerador de Código Máquina:
\begin{verbatim}
%{
    #include <stdio.h>
    #include <stdlib.h>
    #include <string.h>
    #include "hashTable.h"

    int countLabel=1;
    int labelStack[100], sp=0;
    FILE *f;

    HashTable symbolTable;
    char **bloco;
    int i;

    void insertSymbol(char* symb, char* type, int tamanho){

        int res;
        char aux[1000];

        res = hashInsert(symbolTable, symb, type, tamanho);

        if(res == 0){
            sprintf(aux,"Variável '%s' já definida.",symb);
            yyerror(aux);
        }
    }

    int checkSymbol(char* symb){

        int res;
        char aux[1000];

        res = hashContains(symbolTable, symb);

        if(res == 0){
            sprintf(aux,"Variável '%s' não definida.",symb);
            yyerror(aux);
        }
        return res;
    }

    char* checkType(char* symb){

        int res;

        res = hashContains(symbolTable, symb);

        if(res == 0){
            return "ND";
        }
        else{
            return hashType(symbolTable,symb);
        }
    }

    void checkSymbolInit(char* symb){

        int res;

        res = hashIsInit(symbolTable, symb);

    //    printf("Warning linha %d: Variável '%s' não inicializada!\n",symb);
    }

    void initSymbol(char* symb){

        hashInit(symbolTable, symb);
    }

%}

%union {
    int vali;
    char* valc;
    char valOp;
}

%token STRING INT IF ELSE WHILE FOR PRINT INPUT 

%token <valc> id OpComp str
%token <vali> num
%token <valOp> OpA OpM

%type <valc> Var
%type <vali> Exp Termo Fator

%%

Prog    :   Declss Instrs   {fprintf(f,"\tSTOP\n");}
        ;

Declss  :   Decls      {fprintf(f,"\tSTART\n");}

Decls   :   InitVar         {}
        |   Decls InitVar   {}
        ;

InitVar :   INT Var '[' num ']' ';' {
                                        insertSymbol($2,"arrayint",$4);
                                        fprintf(f,"\tPUSHN %d\n", $4);
                                    }
                                        
        |   INT Var ';'             {
                                        insertSymbol($2,"int",0);
                                        fprintf(f,"\tPUSHI 0\n");
                                    }

        |   STRING Var ';'          {
                                        insertSymbol($2,"string",0);
                                        fprintf(f,"\tPUSHS \"\"\n");
                                    }
        ;

Var     :   id              {$$ = $1;}
        ;

Instrs  :   Instr           {}
        |   Instrs Instr    {}
        ;    

Instr   :   If              {}
        |   While           {}
        |   For             {}
        |   Atr ';'         {}
        |   IO ';'          {}
        ;

If      :   IF '(' Cond ')' {
                                labelStack[sp++] = countLabel++;
                                fprintf(f,"\tJZ L%d\n",labelStack[sp-1]);
                            }
            '{' Instrs '}'  Else
        ;

Else    :                   {    
                                fprintf(f,"L%d:\n",labelStack[--sp]);
                            }
        |   ELSE            {
                                fprintf(f,"\tJUMP L%d\n",countLabel);
                                fprintf(f,"L%d:\n",labelStack[--sp]);
                                labelStack[sp++] = countLabel++;
                            } 
            '{' Instrs '}'  {
                                fprintf(f,"L%d:\n",labelStack[--sp]);
                            }
        ;

While   :   WHILE           {
                                labelStack[sp++] = countLabel++;
                                fprintf(f,"L%d:\n",countLabel);
                            }
            '(' Cond ')'    {
                                fprintf(f,"\tJZ L%d\n",labelStack[sp-1]);
                                labelStack[sp++] = countLabel++;
                            }
            '{' Instrs '}'  {
                                fprintf(f,"\tJUMP L%d\n",labelStack[--sp]);
                                fprintf(f,"L%d:\n",labelStack[--sp]);
                            }
        ;

For     :   FOR '(' Atr ';' {
                                labelStack[sp++] = countLabel++;
                                fprintf(f,"L%d:\n",countLabel);
                            }
            Cond ';'        {
                                fprintf(f,"\tJZ L%d\n",labelStack[sp-1]);
                                fprintf(f,"\tJUMP L%d\n",countLabel+2);
                                labelStack[sp++] = countLabel++;
                                fprintf(f,"L%d:\n",countLabel++);
                            }
            Atr ')'         {
                                fprintf(f,"\tJUMP L%d\n", countLabel-2);
                                fprintf(f,"L%d:\n",countLabel++);
                            }
            '{' Instrs '}'  {
                                fprintf(f,"\tJUMP L%d\n",labelStack[--sp]+1);
                                fprintf(f,"L%d:\n",labelStack[--sp]);
                            }
        ;

IO      :   PRINT Out       {}
        |   INPUT Var       {
                                fprintf(f,"\tREAD\n");
                                fprintf(f,"\tATOI\n");
                                fprintf(f,"\tSTOREG %d\n", hashInd(symbolTable,$2));
                            }
        ;

Out     :   Exp             {
                                if($1==1){
                                    fprintf(f,"\tWRITEI\n");
                                }
                                else{
                                    fprintf(f,"\tWRITES\n");
                                }
                            }

        |   str             {
                                fprintf(f,"\tPUSHS %s\n",$1);
                                fprintf(f,"\tWRITES\n");
                            }
        ;

Atr     :   Var '=' Exp {
                                    char aux[1000];
                                if(strcmp(checkType($1),"arrayint")!=0){
                                    if(strcmp(checkType($1),"int")==0){
                                        if($3==1){
                                            if(checkSymbol($1)){
                                                initSymbol($1);
                                                fprintf(f,"\tSTOREG %d\n", 
                                                    hashInd(symbolTable,$1));
                                            }
                                        }
                                        else{
                                            yyerror("Tipos diferentes");
                                        }
                                    }
                                    else if(strcmp(checkType($1),"string")==0){
                                        if($3==2){
                                            if(checkSymbol($1)){
                                                initSymbol($1);
                                                fprintf(f,"\tSTOREG %d\n", 
                                                    hashInd(symbolTable,$1));
                                            }
                                        }
                                        else{
                                            yyerror("Tipos diferentes");
                                        }
                                    }
                                    else if(strcmp(checkType($1),"arrayint")==0){
                                        if($3==1){
                                            if(checkSymbol($1)){
                                                initSymbol($1);
                                                fprintf(f,"\tLOADN\n");
                                            }
                                        }
                                        else{
                                            yyerror("Tipos diferentes");
                                        }
                                    }
                                    else if(strcmp(checkType($1),"ND")==0){
                                        sprintf(aux,"Variável '%s' não definida.",$1);
                                        yyerror(aux);
                                    }
                                }
                                else{
                                    yyerror("Tipos diferentes");
                                }
                        }
        |   Var '[' Exp ']' '=' Exp {
                                        if(strcmp(checkType($1),"arrayint")==0){
                                            fprintf(f,"\tPUSHG %d\n", 
                                                hashInd(symbolTable,$1));
                                            fprintf(f,"\tSTOREN\n");
                                        }
                                        else{
                                            yyerror("Tipos diferentes");
                                        }
                                    }
        ;

Exp     :   Termo           {}
        |   Exp OpA Termo   {
                                if($1 == 1 && $3 == 1){
                                    switch($2){
                                        case '+': 
                                            fprintf(f,"\tADD\n");
                                            break;
                                        case '-': 
                                            fprintf(f,"\tSUB\n");
                                            break;
                                        case '|':
                                            fprintf(f,"\tADD\n");
                                    }
                                }
                                else if($1 == 2 && $3 == 2){
                                    switch($2){
                                        case '+': 
                                            fprintf(f,"\tCONCAT\n");
                                            break;
                                        default:
                                            yyerror("Tipos diferentes");
                                            break;
                                    }
                                }
                                else{
                                    yyerror("Tipos diferentes");
                                }
                            }
        ;

Termo   :   Fator           {$$ = $1;}
        |   Termo OpM Fator {
                                if($1 == 1 && $3 == 1){
                                    switch($2){
                                        case '/': 
                                            fprintf(f,"\tDIV\n");
                                            break;
                                        case '*': 
                                            fprintf(f,"\tMUL\n");
                                            break;
                                        case '%': 
                                            fprintf(f,"\tMOD\n");
                                            break;
                                        case '&':
                                            fprintf(f,"\tMUL\n");
                                            break;
                                    }
                                }
                                else{
                                    yyerror("Tipos diferentes");
                                }
                            }
        ;

Fator   :   Var     {
                                if(checkSymbol($1))
                                    checkSymbolInit($1);

                                if(strcmp(checkType($1),"int")==0){
                                    $$=1;
                                    fprintf(f,"\tPUSHG %d\n", hashInd(symbolTable,$1));
                                }
                                else if(strcmp(checkType($1),"string")==0){
                                    $$=2;
                                    fprintf(f,"\tPUSHG %d\n", hashInd(symbolTable,$1));
                                }
                                else{
                                }
                    }

        |   Var '[' Exp ']'     {
                                    if(strcmp(checkType($1),"arrayint")==0){
                                        $$=1;
                                        printf("%s\n",checkType($1));
                                        fprintf(f,"\tPUSHG %d\n", hashInd(symbolTable,$1));
                                        fprintf(f,"\tLOADN\n");
                                    }
                                    else{
                                        yyerror("Tipos diferentes");
                                    }
                                }
        |   num             {$$ = 1; fprintf(f,"\tPUSHI %d\n", $1);}
        |   str             {$$ = 2; fprintf(f,"\tPUSHS %s\n", $1);}
        |   '(' Exp ')'     {}
        |   '!' Exp         {}
        ;

Cond    :   Comp                {}
        |   '(' Cond ')'        {}
        |   Cond '&''&' Cond    {
                                    fprintf(f,"\tMUL\n");
                                }
        |   Cond '|''|' Cond    {
                                    fprintf(f,"\tADD\n");
                                }
        ;

Comp    :   Exp             {}
        |   Exp OpComp Exp  {
                                switch($2[0]){
                                    case '>':
                                        if($2[1] == '='){
                                            fprintf(f,"\tSUPEQ\n");
                                        }
                                        else{
                                            fprintf(f,"\tSUP\n");
                                        }
                                        break;
                                    case '<':
                                        if($2[1] == '='){
                                            fprintf(f,"\tINFEQ\n");
                                        }
                                        else{
                                            fprintf(f,"\tINF\n");
                                        }
                                        break;
                                    case '=':
                                        fprintf(f,"\tEQUAL\n");
                                        break;
                                    case '!':
                                        fprintf(f,"\tEQUAL\n");
                                        fprintf(f,"\tNOT\n");
                                        break;

                                }
                            }
        ;

%%

#include "lex.yy.c"

int yyerror(char *s){
    printf("Erro Sintático linha %d: %s\n",yylineno, s);
}
     
int main(){

    symbolTable = hashCreate(1000);
    bloco = malloc(sizeof(char*)*1000);
    f = fopen("assembly","w");
    yyparse(); 
    return 0; 
}

\end{verbatim}

\end{document} 
