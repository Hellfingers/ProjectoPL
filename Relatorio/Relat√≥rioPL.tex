%
% Layout retirado de http://www.di.uminho.pt/~prh/curplc09.html#notas
%
\documentclass{report}
\usepackage[portuges]{babel}
\usepackage[utf8]{inputenc}
\usepackage{url}
%\usepackage{alltt}
%\usepackage{fancyvrb}
\usepackage{listings}
%LISTING - GENERAL
\lstset{
	basicstyle=\small,
	numbers=left,
	numberstyle=\tiny,
	numbersep=5pt,
	breaklines=true,
    frame=tB,
	mathescape=true,
	escapeinside={(*@}{@*)}
}
%
%\lstset{ %
%	language=Java,							% choose the language of the code
%	basicstyle=\ttfamily\footnotesize,		% the size of the fonts that are used for the code
%	keywordstyle=\bfseries,					% set the keyword style
%	%numbers=left,							% where to put the line-numbers
%	numberstyle=\scriptsize,				% the size of the fonts that are used for the line-numbers
%	stepnumber=2,							% the step between two line-numbers. If it's 1 each line
%											% will be numbered
%	numbersep=5pt,							% how far the line-numbers are from the code
%	backgroundcolor=\color{white},			% choose the background color. You must add \usepackage{color}
%	showspaces=false,						% show spaces adding particular underscores
%	showstringspaces=false,					% underline spaces within strings
%	showtabs=false,							% show tabs within strings adding particular underscores
%	frame=none,								% adds a frame around the code
%	%abovecaptionskip=-.8em,
%	%belowcaptionskip=.7em,
%	tabsize=2,								% sets default tabsize to 2 spaces
%	captionpos=b,							% sets the caption-position to bottom
%	breaklines=true,						% sets automatic line breaking
%	breakatwhitespace=false,				% sets if automatic breaks should only happen at whitespace
%	title=\lstname,							% show the filename of files included with \lstinputlisting;
%											% also try caption instead of title
%	escapeinside={\%*}{*)},					% if you want to add a comment within your code
%	morekeywords={*,...}					% if you want to add more keywords to the set
%}

\usepackage{xspace}

\parindent=0pt
\parskip=2pt

\def\darius{\textsf{Darius}\xspace}
\def\java{\texttt{Java}\xspace}

\title{Processamento de Linguagens (3º ano de Curso)\\ \textbf{Trabalho Prático 2}\\ Relatório de Desenvolvimento}
\author{Jorge Miguel Sol Ferreira (a64293) \and Pedro José Freitas da Cunha (a67677) \and José Pedro Brito Pereira (a67680)\\Grupo 35 }
\date{\today}

\begin{document}

\maketitle

\begin{abstract}
Este relatório documentará todos os passos tomados na realização do segundo trabalho prático da Unidade Curricular de Processamento de Linguagens.\\ Neste projecto é requerida a implementação de um compilador de uma Linguagem de Programação Imperativa Simples e posteriormente gerador de código assembly para uma máquina de stacks virtual.
\end {abstract}
\tableofcontents

\chapter{Introdução} \label{intro}

\begin{description}
  \item [Enquadramento] Um \textbf{compilador} é uma peça de software que transforma o código fonte numa dada linguagem de alto nível em instruções que a máquina entenda (Código Máquina). As fazes da compilação incluem:\\Análise léxica\\Análise Sintática\\Análise Semântica\\Geração de Código
  \item [Conteúdo do documento] Neste documento encontrar-se-ão as fases de resolução do problema especificado
  \item [Resultados -- pontos a evidenciar] O resultado do projecto a desenvolver será um gerador de códigoassembly para uma máquina de stacks virtual, partindo de uma Linguagem de Programação Imperativa Simples.
\end{description}



\section*{Estrutura do Relatório} \
No capítulo \ref{analiseEsp} iremos apresentar o caso de estudo em causa. No capítulo \ref{concepcao} iremos apresentar a estrutura de dados auxiliar à análise semântica e sua utilização no compilador a desenvolver, bem como fazer um esboço do que queremos que seja a nossa linguagem de programação (Gramática Independente do Contexto), para além de alguns exemplos de frases que tenham erros sintáticos, semânticos e frases correctas segundo a nossa especificação. No capítulo \ref{code} iremos apresentar os passos utilizados para geração de código, se possível ou, em alternativa notificaçao de erro sintático. Finalmente, no capítulo \ref{conc} faremos uma apreciação crítica do trabalho realizado e trabalhos futuros. Em anexo iremos colocar o código desenvolvido que permitirá a geração de código máquina.

\chapter{Análise e Especificação} \label{analiseEsp}

\section{Descrição informal do problema}
Neste projecto é pretendido o desenho de uma Linguagem de Programação Imperativa Simples, para de seguida criar um compilador que gere pseudo-código Assembly de uma Máquina Virtual de Stacks
\section{Especificação do Requisitos}
Os requisitos para o compilador/linguagem a implementar são os seguintes:
\begin{itemize}
  \item Permitir manusear variáveis do tipo inteiro(escalar ou array).
  \item Realizar as seguintes operações:
	\begin{itemize}
	\item Atribuições de expressões a variáveis.
	\item Ler do Standard Input.
	\item Escrever para o Standard Output.
	\end{itemize}
 \item Ciclos(for, while) e instruções Condicionais(if..else).
 \item Operações Aritméticas, Relacionais e Lógicas sobre inteiros.
 \item Indexação sobre arrays.
 \item As declarações de variáveis deverão ser no início do programa.
 \item Não deverá ser possível realizar redeclaraçoes nem utilizações sem declaraçao prévia.
 \item Se não existirem atribuições a uma variável, o valor da mesma deverá ser indefinido
\end{itemize}

\chapter{Concepção/desenho da Resolução} \label{concepcao}
\section{Gramática}
Ao desenvolver a gramática tentámos fazer com que a mesma ficasse o mais próximo possível do C.\\
%GIC
\section{Estruturas de Dados}
Para realizarmos a análise semântica temos uma tabela de Hash para guardar todos os identificadores de variáveis\ldots


\chapter{Codificação e Testes} \label {code}
\section{Alternativas, Decisões e Problemas de Implementação}
\section{Testes realizados e Resultados}
Mostram-se a seguir alguns testes feitos (valores introduzidos) e
os respectivos resultados obtidos:

%\VerbatimInput{teste1.txt}


\chapter{Conclusão} \label {conc}
Síntese do Documento.\\
Estado final do projecto; Análise crítica dos resultados.\\
Trabalho futuro.

\appendix
\chapter{Código do Programa}


\begin{verbatim}
      aqui deve aparecer o código do programa,
      tal como está formato no ficheiro-fonte "darius.java"
\end{verbatim}

















\end{document} 